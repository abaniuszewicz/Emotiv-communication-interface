% !TEX encoding = UTF-8 Unicode
% !TeX spellcheck = pl_PL
% !TEX TS-program = xelatex
% !TeX TXS-program:bibliography = txs:///biber
% !TeX TXS-program:index = txs:///makeindex

\documentclass[skorowidz,skroty]{dyplomWEZUT}
% ------------------------------------------------------------------------------
% Opcje klasy <dyplomWEZUT>
% 1) skorowidz - włącza skorowidz
% 2) skroty    - włącza wykaz ważniejszych skrótów i oznaczeń
% ------------------------------------------------------------------------------


% -------------------------- Dane pracy dyplomowej -----------------------------

% Imię i Nazwisko
\author{Adam Baniuszewicz}

% Numer albumu
\nralbumu{33816}

% Tytuł pracy
\title{Algorytmy poleceń mentalnych w~interfejsach mózg-komputer}

% Tytuł pracy w języku angielskim
\tytulang{Algorithms of~mental commands in~brain-computer interfaces}

% Kierunek studiów
\kierunek{Teleinformatyka}

% Rodzaj studiów: S1/S2/N1/N2
\rodzajstudiow{S2}

% Specjalność (tylko na studiach magisterskich - S2 i N2)
\specjalnosc{Sieci teleinformatyczne i~systemy mobilne}

% Data wydania tematu w SIWE/eDziekanacie
\datawydania{01.11.2018}

% Data złożenia pracy w eDziekanacie
\datazlozenia{TODO}

% Miejsce złożenia pracy (odkomentować i wypełnić jeżeli inne niż Szczecin)
%\miejsce{}

% Imię i nazwisko opiekuna - wpisujemy w dopełniaczu
\opiekun{dr. inż. Roberta Krupińskiego}

% Katedra, Zakład lub Zespół Dydaktyczny
% (Wydział wpisujemy po przecinku tylko jeśli inny niż WE)
\jednostka{Katedra Przetwarzania Sygnałów i~Inżynierii Multimedialnej}

% Słowa kluczowe
\slowakluczowe{TODO, TODO}

% Słowa kluczowe po angielsku
\keywords{TODO, TODO}

%% ----------------------- Koniec definicji danych -----------------------------

% Dodanie metadanych do wynikowego pdf (Autor, Tytuł, Słowa kluczowe)
\makemetadata

\begin{document}

\begin{streszczenie}
TODO
\end{streszczenie}

\begin{abstract}
TODO
\end{abstract}

\maketitle

\begin{wprowadzenie}

TODO

\end{wprowadzenie}

\cel{TODO}

\zakres{TODO}

\chapter{Analiza dostępnych rozwiązań BCI}


\begin{zakonczenie}\label{chap:zakonczenie}
TODO
\end{zakonczenie}

% Bibliografia
\printbibliography[heading=bibintoc]

% Spis tabel (jeżeli jest potrzebny)
\listoftables

% Spis rysunków (jeżeli jest potrzebny)
\listoffigures

% Spis kodów źródłowych (jeżeli jest potrzebny)
\listoflistings

% TODO: Przenieść generowanie do pakietu glossaries i wyemilinować potrzebę manualnego wykonywania polecenia makeindex w terminalu. !! Dyskusyjne, indeksy należy definiować w preambule lub w osobnym pliku (tak samo jak acronyms), ale pozwala na większą swobodę w definiowaniu wpisów i nie wymaga wywoływania makeindex !!

% Skorowidz (opcjonalnie), po skompilowaniu dokumentu należy użyć opcji Narzędzia -> Indeks,
% aby wygenerować wpisy, po czym powtórnie skompilować dokument.
\printindex

\end{document}
