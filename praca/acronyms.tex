% ------------------------------------------------------------------------------
% Baza definicji skrótów i oznaczeń.
% Importowana automatycznie na poziomie klasy
% ------------------------------------------------------------------------------
% Reguły użycia opcji sort (aby spełnić wymagania sortowania):
% -- skróty poprzedzamy znakiem plus (+)
% -- symbole łacińskie poprzedzamy podkreślnikiem (_)
%    i używamy wielkości liter tożsamych z symbolem
% -- dla symboli z literami greckimi podajemy ich nazwę łacińską,
%    np. α -> alfa oraz ustawiamy pierwszą literę na wielką,
%    jeżeli symbol grecki jest także pisany wielką literą,
%    np. Γ -> Gamma
% ------------------------------------------------------------------------------
%\newacronym[sort=+fft]{fft}{FFT}{Fast Fourier Transform --- Szybka Transformata Fouriera}
%\newacronym[sort=+ifft]{ifft}{IFFT}{Inverse Fast Fourier Transform --- Odwrotna Szybka Transformata Fouriera}
%\newacronym[sort=_Hs]{hs}{$H(S)$}{Transmitancja}
%\newacronym[sort=_ht]{ht}{$h(t)$}{Okno czasowe}
%\newacronym[sort=alfa]{alfa}{$α$}{Współczynnik rozszerzalności termicznej}

\newacronym[sort=bci]{bci}{BCI}{Brain--computer interface --- Interfejs mózg-komputer}
\newacronym[sort=eeg]{eeg}{EEG}{Elektroencefalografia}
\newacronym[sort=emg]{emg}{EMG}{Elektromiografia}
\newacronym[sort=eog]{eog}{EOG}{Elektrookulografia}
\newacronym[sort=ecg]{ecg}{ECG}{Elektrokardiografia}
\newacronym[sort=api]{api}{API}{Application programming interface --- Interfejs programistyczny aplikacji}
\newacronym[sort=sdk]{sdk}{SDK}{Software development kit --- Zestaw narzędzi do tworzenia oprogramowania}